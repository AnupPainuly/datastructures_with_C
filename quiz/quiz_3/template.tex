\documentclass{report}
\input{preamble}
\input{macros}
\input{letterfonts}

\title{\Huge{Quiz Three\\Datastructures}\\Linked List}
\author{\huge{Smita Kadam}}
\date{18-Oct-2022}

\begin{document}
\maketitle
\qs{}{
Below one is add_first() functionality code for empty Doubly CircularLinked list. Find the correct statement to replace ????
	head = newnode;
	tail = newnode;
	tail ->next = head;
	????
}
\setlength{\parindent}{0pt}
Options:\\
1. tail->next->next = tail
2. \underline{head->prev = tail;}
3. tail->prev = head;
4. head->prev = NULL;
\qs{}{
what does following function do of given Linked List with first node as head?
void function(struct node* head)
}
\setlength{\parindent}{0pt}
Options:\\
1. Print all linked lists
2. \underline{Prints all linked list in reverse order}
3. Prints alternate Linked List
4. Prints alternate linked list reverse order
\qs{}{
what is wrong about singly linked list?
}
\setlength{\parindent}{0pt}
Options:\\
1. Singly linked list is a collection of nodes linked together in a sequential way where each node of singly linked list contains a data field
2. singly list is an address field which contains the reference of the next node.
3. Singly linked list can contain multiple data fields but should contain at least single address field pointing to its connected next node.
4. \underline{None Of Above}
\qs{
In a singly circular linked list, insertion of a node at first position requires modification of a?
}
\setlength{\parindent}{0pt}
Options:\\
1. One pointer
2. \underline{Two pointer}
3. Three pointer
4. None

\qs{}{
following code define in c programming language
struct node
{
int data;
struct node * next;
}
typedef struct node NODE;
NODE *ptr;
Which of the following c code is used to create new node?
}
\setlength{\parindent}{0pt}
1. \underline{ptr=(NODE*)malloc(sizeof(NODE));}
2. ptr=(NODE*)malloc(NODE);
3. ptr=(NODE*)malloc(sizeof(NODE*));
4. ptr=(NODE)malloc(sizeof(NODE));
\qs{}{
Which of the following is false about a doubly linked list?
}
\setlength{\parindent}{0pt}
Options:\\
1. We can navigate in both the directions
2. It requires more space than a singly linked list
3. The insertion and deletion of a node take a bit longer
4. \underline{Implementing a doubly linked list is easier than singly linked list}
\pagebreak
\qs{}{
which statement is True about circular linked list?
}
\setlength{\parindent}{0pt}
Options:\\
1. Entire list can be traversed from any node.
2. Circular lists are the required data structure when we want a list to be accessed in a circle or loop.
3. Despite of being singly circular linked list we can easily traverse to its previous node, which is not possible in singly linked list.
4. \underline{All of Above}

\qs{}{
Which of the following data structure/s requires extra space to store n no. of elements?
}
\setlength{\parindent}{0pt}
Options:\\
1. Array
2. Structure
3. \underline{LinkedList}
4. Union
\qs{}{
How do you calculate the pointer difference in a memory efficient doubly linked list?
}
\setlength{\parindent}{0pt}
Options:\\
1. head xor tail
2. \underline{pointer to previous node xor pointer to next node}
3. pointer to previous node – pointer to next node
4. pointer to next node – pointer to previous node
\qs{}{
Consider the following doubly linear linked list and find the output of given code: 
	head								tail	
	1	<-->	2	<-->	3	<-->	4	<-->	5
        4000		2000		2800		4800		3000

trav= tail;
while(trav!=NULL && trav->prev!=NULL)
{
   print("%d-->",trav->data);
   trav = trav->prev->prev;		
}
}
\setlength{\parindent}{0pt}
Options:
1. 5-->3-->1
2. 5-->4-->3-->2-->1
3. \underline{5-->3-->}
4. 1-->3-->5
\setlength{\parindent}{0pt}
\newpage% or \cleardoublepage
% \pdfbookmark[<level>]{<title>}{<dest>}
\pdfbookmark[section]{\contentsname}{toc}
%\tableofcontents
\pagebreak
\end{document}
